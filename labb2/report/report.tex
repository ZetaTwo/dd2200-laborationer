\documentclass[10pt,a4paper]{article}
\usepackage[utf8]{inputenc}
\usepackage[T1]{fontenc}
\usepackage{amsmath}
\usepackage{amsfonts}
\usepackage{amssymb}

\usepackage{listings}
\usepackage{color}
\usepackage{tikz}
\usetikzlibrary{arrows}
\usepackage{hyperref}

\title{DD2200 - Operativsystem \\ Laboration 2 \\ Small-Shell v2.51 för UNIX \\ Period 4, läsår 2014}
\author{Carl Svensson, F-10 \\ 910414-1412 \\ carlsven@kth.se}
\date{}

\definecolor{dkgreen}{rgb}{0,0.6,0}
\definecolor{gray}{rgb}{0.5,0.5,0.5}
\definecolor{mauve}{rgb}{0.58,0,0.82}

\lstdefinestyle{cstyle}{%
  language=C,
  aboveskip=3mm,
  belowskip=3mm,
  showstringspaces=false,
  columns=flexible,
  basicstyle={\small\ttfamily},
  numbers=none,
  numberstyle=\color{mauve},
  keywordstyle=\color{blue},
  commentstyle=\color{dkgreen},
  stringstyle=\color{mauve},
  breaklines=true,
  breakatwhitespace=true
  tabsize=3
}

\lstdefinestyle{bashstyle}{%
  language=C,
  aboveskip=3mm,
  belowskip=3mm,
  showstringspaces=false,
  columns=flexible,
  basicstyle={\small\ttfamily},
  numbers=none,
  numberstyle=\color{mauve},
  keywordstyle=\color{blue},
  commentstyle=\color{dkgreen},
  stringstyle=\color{mauve},
  breaklines=true,
  breakatwhitespace=true
  tabsize=3
}

\lstset{
  style=bashstyle
}

\begin{document}

\maketitle
\tableofcontents
\clearpage

\section{Problembeskrivning}

Laborationen går ut på att skriva ett litet skal för att köra kommandon. Skalet ska ha två inbygda kommandon: exit och cd. Exit avslutar skalet och cd låter användaren byta den aktuella katalogen. Alla andra kommandon körs genom att skapa en barnprocess som kör det aktuella programmet. Program ska gå att starta både i för- och bakgrunden och användaren ska meddelas om förgrundsprogrammens körtid. Användaren ska också meddelas när bakgrundsprocesser avslutas och detta ska göras vid nästa kommandoinmatning. Vid keyboard-interrupts så ska enbart den barnprocess som körs i förgrunden avslutas och inte själva skalet.

\subsection{Förberedelsefrågor}

\begin{enumerate}
\item Till att börja med är det mycket lättare att utnyttja redan existerande program utan att det blir ett problem att komma tillbaka till kommandoinmatningen. Det blir också mycket lättare att återställa till ursprungsläget när processen är klar. Om processen t.ex. dirigerar om \emph{stdin} eller \emph{stdout} så slängs dessa ändringar bort när processen avslutar. Dessutom så skyddar det lite mot minnesläckor. Om ett program som körs i en separat process läcker minne så kommer det ändå städas undan när processen avslutas vilket antagligen är mycket tidigare än när hela skalet avslutas.
\item Processen kommer att ligga kvar och ta upp resurser men inte längre köras. Den kommar att vara en s.k. zombie-process. Det är därför viktigt att anropa \emph{wait()} så att alla resurser frigörs ordentligt.
\item \emph{SIGSEGV} står för "segmentation violation" vilket oftast kallas för "segmentation fault".
\item Från ett designperspektiv så är det rimligt att operativsystemet måste kunna avsluta en process oavsett om processen själv vill det eller inte. Från programmerarens perspektiv så går det helt enkelt inte att anropa \emph{sigaction} med \emph{SIGKILL} som argument.
\item "void (*disp)(int)" är en variabel med namnet "disp" som är en funktionspekare till en funktion som tar en int som argument och inte returnerar något (void).
\item Från man-sidan till \emph{sigset}: "...compatibility interface..." och "new applications should use ... \emph{sigaction(2)}, \emph{sigprocmask(2)}" vilket då antagligen är de funktioner som \emph{sigset(3)} själv använder.
\item Installera en egen signalhanterare som tar hand om signalen \emph{SIGINT} vilket är "keyboard interrupt" och ignorerar eller skickar vidare den.
\item Varje process har sin egen working directory så när \emph{cd} körs i miniShell-processen så påverkar inte det working directory i processen som startade miniShell. Det gör att när sedan miniShell avslutas med "exit" så återgår körningen till den startande processen vars working directory är oförändrad.
\end{enumerate}

\section{Programbeskrivning}

Programmet kör en huvudloop. Varje varv av loopen läser in ett kommando från användaren. Om kommandot är \emph{exit} så avslutas loopen och därmed programmet. Om det är \emph{cd} så försöker programmet byta den nuvarande arbetskatalogen. I annat fall så försöker programmet starta det program som användaren matat in med tillhörande argument. Detta görs genom att programmet \emph{fork}ar och barnprocessen använder \emph{exec} för att köra kommandot. Om det är en förgrundsprocess så tar förälderprogrammet tid på hur lång tid barnprocessen körs och väntar tills dess att den kört klart. Om det är en bakgrundsprocess så läggs process-ID:t till listan över nuvarande bakgrundsprocesser.
Efter att ett kommando har körts så gås listan över bakgrundsprocesser igenom för att kontrollera om någon eller några av dessa har kört klart. Detta meddelas i så fall och plockas då bort från listan.

\clearpage
\subsection{Testkörning}

\begin{lstlisting}
/cygdrive/f/Dropbox/Skola/kth/os/labbar/labb2> cd ..
/cygdrive/f/Dropbox/Skola/kth/os/labbar> ls
Installing
Spawned foreground process pid: 2836
labb1                  labb1.sublime-workspace  labb3
labb1.sublime-project  labb2                    labb-pm.pdf
Restoring
Foreground process 2836 terminated
Wallclock time: 6.460000 ms.
/cygdrive/f/Dropbox/Skola/kth/os/labbar> ping google.com
Installing
Spawned foreground process pid: 7648

Pinging google.com [173.194.32.8] with 32 bytes of data:
Reply from 173.194.32.8: bytes=32 time=1ms TTL=55
...
Reply from 173.194.32.8: bytes=32 time=1ms TTL=55

Ping statistics for 173.194.32.8:
    Packets: Sent = 4, Received = 4, Lost = 0 (0% loss),
Approximate round trip times in milli-seconds:
    Minimum = 1ms, Maximum = 1ms, Average = 1ms
Restoring
Foreground process 7648 terminated
Wallclock time: 3023.795000 ms.
/cygdrive/f/Dropbox/Skola/kth/os/labbar> ping google.com &
Installing
Spawned background process pid: 2384
Restoring
Background process 2384 terminated
/cygdrive/f/Dropbox/Skola/kth/os/labbar>
Pinging google.com [173.194.32.8] with 32 bytes of data:
Reply from 173.194.32.8: bytes=32 time=1ms TTL=55
ls
Installing
Spawned foreground process pid: 5072
labb1  labb1.sublime-project  labb1.sublime-workspace  labb2  labb3  labb-pm.pdf
Restoring
Foreground process 5072 terminated
Wallclock time: 6.791000 ms.
/cygdrive/f/Dropbox/Skola/kth/os/labbar> Reply from 173.194.32.8: bytes=32 time=1ms TTL=55
Reply from 173.194.32.8: bytes=32 time=1ms TTL=55
Reply from 173.194.32.8: bytes=32 time=1ms TTL=55

Ping statistics for 173.194.32.8:
    Packets: Sent = 4, Received = 4, Lost = 0 (0% loss),
Approximate round trip times in milli-seconds:
    Minimum = 1ms, Maximum = 1ms, Average = 1ms

/cygdrive/f/Dropbox/Skola/kth/os/labbar>

\end{lstlisting}


\clearpage
\section{Källkod}
Källkoden går att läsa nedan. Den går också att ladda ner från Github\footnote{\url{https://github.com/ZetaTwo/dd2200-laborationer/tree/master/labb2}} och kompileras med "make"

\subsection{Smallshell.c}
\lstinputlisting[style=cstyle]{../src/smallshell.c}
\subsection{Timer.c}
\lstinputlisting[style=cstyle]{../src/timer.c}
\subsection{Command.c}
\lstinputlisting[style=cstyle]{../src/command.c}
\subsection{Processlist.c}
\lstinputlisting[style=cstyle]{../src/processlist.c}
\clearpage

\section{Arbetsgång och Utvärdering}

Notera: Vissa saker jag skriver här är i stort sett samma sak som i labb 1.

Labben och rapporten har tillsammans tagit ca. 7 timmar. Materialet som fanns att tillgå var rätt bra. Däremot känns det som att det finns ett stort överlapp mellan innehållet i de olika labb-PM och det allmänna labb-PM:et. Jag har blivit mycket bättre införstådd i de systemanrop som används och hur signaler, fork och exec fungerar.

Mitt arbetssätt gick i stora drag till enligt följande. Jag började med att skapa funktioner för att läsa in och bearbeta ett kommando och lägga in detta i en loop. Där implementerades sedan switchen som väljer vad som ska göras med kommandot. Exit-funktionen implemnterades trivialt med ett \emph{break}.
Sedan skapade jag funktionen som byter arbetskatalog. Här behövde jag ändra lite i hur jag bearbetar kommandon så att jag behåller med den "tokeniserade" och ursprungliga texten eftersom jag ville kunna göra \emph{cd} med kataloger som innehöll mellanslag. Det behövdes även göras skillnad på sökvägar som börjar med "/" och övriga sökvägar.
Efter detta implementerade jag själva kärnan i programmet, dvs den del som \emph{fork}ar och kör kommandon. Här utgick jag mycket från kunskaperna i labb 1 och det gick därför mycket snabbt att implementera.
Jag la sedan till signalhanteraren som gjorde att signaler om avslut endast berörde barnprocessen om denna kördes.
Slutligen implementerade jag en enkelt länkad lista som höll koll på bakgrundsprocesser och att denna kollades igenom efter varje kört kommando.

Jag skulle säga att materialet är bra men överlappet gör att man ibland tappar fokus när man läser "det där har jag redan läst" så att man riskerar att missa saker. Det skulle kanske vara bra att flytta ut mer material till ett kompendium om systemanrop, etc. som är ett förkunskapskrav för både labb 1 och 2.

Gällande såvrighet så är jag rätt bekant med C (egentligen C++) så den biten var inte så farlig men vissa systemanrop och att strukturera det bra var lite småklurigt ändå. Kortfattat:

\begin{itemize}
\item Labb-PM: 3
\item Svårighetsgrad: 3
\item Lärorikhet: 4
\end{itemize}



\end{document}