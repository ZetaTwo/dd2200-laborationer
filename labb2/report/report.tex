\documentclass[10pt,a4paper]{article}
\usepackage[utf8]{inputenc}
\usepackage[T1]{fontenc}
\usepackage{amsmath}
\usepackage{amsfonts}
\usepackage{amssymb}

\usepackage{listings}
\usepackage{color}
\usepackage{tikz}
\usetikzlibrary{arrows}
\usepackage{hyperref}

\title{DD2200 - Operativsystem \\ Laboration 2 \\ Small-Shell v2.51 för UNIX \\ Period 1, läsår 2014}
\author{Carl Svensson, F-10 \\ 910414-1412 \\ carlsven@kth.se}
\date{}

\definecolor{dkgreen}{rgb}{0,0.6,0}
\definecolor{gray}{rgb}{0.5,0.5,0.5}
\definecolor{mauve}{rgb}{0.58,0,0.82}

\lstdefinestyle{cstyle}{%
  language=C,
  aboveskip=3mm,
  belowskip=3mm,
  showstringspaces=false,
  columns=flexible,
  basicstyle={\small\ttfamily},
  numbers=none,
  numberstyle=\color{mauve},
  keywordstyle=\color{blue},
  commentstyle=\color{dkgreen},
  stringstyle=\color{mauve},
  breaklines=true,
  breakatwhitespace=true
  tabsize=3
}

\lstdefinestyle{bashstyle}{%
  language=C,
  aboveskip=3mm,
  belowskip=3mm,
  showstringspaces=false,
  columns=flexible,
  basicstyle={\small\ttfamily},
  numbers=none,
  numberstyle=\color{mauve},
  keywordstyle=\color{blue},
  commentstyle=\color{dkgreen},
  stringstyle=\color{mauve},
  breaklines=true,
  breakatwhitespace=true
  tabsize=3
}

\lstset{
  style=bashstyle
}

\begin{document}

\maketitle
\tableofcontents
\clearpage

\section{Problembeskrivning}



\subsection{Förberedelsefrågor}

\begin{enumerate}
\item 
\end{enumerate}

\section{Programbeskrivning}



\subsection{Testkörning}

\begin{lstlisting}
$ digenv.exe HOME

> HOME=/home/calle_000
\end{lstlisting}


\clearpage
\section{Källkod}
Källkoden går att läsa nedan. Den går också att ladda ner från Github\footnote{\url{https://github.com/ZetaTwo/dd2200-laborationer/tree/master/labb2}} och kompileras med "make"

\subsection{Smallshell.c}
\lstinputlisting[style=cstyle]{../src/smallshell.c}
\subsection{Timer.c}
\lstinputlisting[style=cstyle]{../src/timer.c}
\subsection{Command.c}
\lstinputlisting[style=cstyle]{../src/command.c}
\clearpage

\section{Arbetsgång och Utvärdering}




\end{document}